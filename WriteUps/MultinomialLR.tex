\documentclass[12pt]{article}
\usepackage[utf8]{inputenc}
\usepackage[margin=0.75in]{geometry}
\usepackage{amsmath}
\usepackage{amssymb}
\usepackage{amsthm}
\usepackage{graphicx} % Required for inserting images

\title{Binary Logistic Regression}
\author{Aaron Avram}
\date{June 4 2025}

\begin{document}

\maketitle

\section*{Introduction}
In this write up I will go through my derivation of the objective function and the optimization algorithm for my Multinomial
Logistic Regression model.

\section*{Construction}
For my notation let $X \in \mathbb{R}^{n \times p}$ be the matrix
with each of the n training inputs as rows. Where we define p to be the dimension
of each input vector (or equivalently the number of features). Next
let $y \in \{1, \ldots, K \}^n$ equal the vector of training labels, where there are K
output classes. I will apply the shortcut used in the Binary Model Construction
where each row of X is given an additional final entry with a 1, to account for the bias term. So $X$
is now a $n \times (p+1)$ dimensional matrix. Where we can let $p + 1 = d$ as our feature number. The model is parametrized by $K - 1$
$p + 1$ dimensional vectors $\theta^{(1)}, \ldots, \theta^{(k-1)}$, which I will collectively refer to by the
flattened combined vector $\theta$. Where the probabilities
are given by:
\begin{align*}
    \Pr(\kappa | x; \theta^{(\kappa)}) &= \begin{cases}
        \frac{\exp((\theta^{(\kappa)})^Tx)}{1 + \displaystyle \sum_{1 \leq i < K}\exp((\theta^{(i)})^Tx)} & \kappa < K \\
        \frac{1}{1 + \displaystyle \sum_{1 \leq i < K}\exp((\theta^{(i)})^Tx)} & \kappa = K
    \end{cases}
\end{align*}
The log likelihood function is then given by:
\begin{align*}
    \mathcal{L}(\theta) &= \displaystyle \sum_{1 \leq r \leq n}\log\Pr(y_r | x_r; \theta^{y_r}) \\
    &\text{Using a one hot encoding with the kronecker delta function we can write this as: } \\
    &= \displaystyle \sum_{1 \leq r \leq n} \sum_{1 \leq s \leq K}\delta_{y_rs}\log\Pr(y_r | x_r; \theta^{(s)}) \\
\end{align*}
Consider the Likelihood function with respect to one input vector $x_r$
and one parameter $\theta^{(j)}$:
\begin{align*}
    \mathcal{L}_{rj}(\theta) &= \log\Pr(y_r | x_r)  
\end{align*}
However we can condense this further, as if $y_r \neq j$ the numerator of the probability
function is irrelevant to $\theta^{(j)}$, and if we decompose the fraction via the logarithm rules
we can omit the numerator term and replace it with a kronecker delta. I.e Suppose $\Pr = N/D$
then $\log(\Pr) = \log(N) - \log(D)$  and so if N is independent of $\theta^{(j)}$ we can omit it.
Thus:
\begin{align*}
     \mathcal{L}_{rj}(\theta) &= \delta_{{y_r}j}\log(\exp((\theta^{(j)})^Tx_r)) - \log(1 + \displaystyle \sum_{1 \leq i < K}\exp((\theta^{(i)})^Tx)) \\
     &= \delta_{{y_r}j}(\theta^{(j)})^Tx_r - \log(1 + \displaystyle \sum_{1 \leq i < K}\exp((\theta^{(i)})^Tx))
\end{align*}

\section*{Gradient Calculations}
First lets take the partial derivative with respect to $\theta^{(j)}$ we find
\begin{align*}
    \frac{\partial \mathcal{L}_{rj}}{\partial \theta^{(j)}} &= \delta_{y_r j}x_r - \frac{x_r\exp((\theta^{(j)})^Tx_r)}{1 + \displaystyle \sum_{1 \leq i < K}\exp((\theta^{(i)})^Tx)} \\
    &= x_r(\delta_{y_rj} - \Pr(j|x_i))
\end{align*}
Reusing some calculations from the previous write up it is clear that
$\frac{\partial \mathcal{L}}{\partial \theta^{(j)}} = X(y - p_j)$ Which form 
the columns of the Jacobian matrix of $\mathcal{L}$.


Now let us take the second partial derivative of this expression with respect to $\theta^{(i)}$
\begin{align*}
    \frac{\partial}{\partial \theta^{(i)}}\frac{\partial \mathcal{L}_{rj}}{\partial \theta^{(j)}} &= \begin{cases}
        -x_rx_r^T\Pr(j|x_r)(1 - \Pr(j|x_r)) & i = j \\
        x_rx_r^T\Pr(j|x_r)\Pr(i|x_r) & i \neq j
    \end{cases}
\end{align*}
We can make this more compact utilizing the kronecker delta function:
\begin{align*}
    \frac{\partial}{\partial \theta^{(i)}}\frac{\partial \mathcal{L}_{rj}}{\partial \theta^{(j)}} &=
    x_rx_r^T\Pr(i|x_r)(\delta_{ij} - \Pr(j|x_r))
\end{align*}
Now let $\Pr(j|x_r) = p_{jr}$ So the total second derivative of $\mathcal{L}$ is:
\begin{align*}
    \frac{\partial}{\partial \theta^{(i)}}\frac{\partial \mathcal{L}}{\partial \theta^{(j)}} &=
    \displaystyle \sum_{1 \leq r \leq n}x_rx_r^Tp_{ir}(\delta_{ij} - p_{jr}) \\
    &= \displaystyle \sum_{1 \leq r \leq n}x_rx_r^Tp_{ir}(\delta_{ij} - p_{jr})
\end{align*}
If we let $W_{ij} = \text{diag}(p_{ir}(\delta_{ij} - p_{jr}))_r$ We can write this more compactly by observing that
\begin{align*}
    \frac{\partial}{\partial \theta^{(i)}}\frac{\partial \mathcal{L}}{\partial \theta^{(j)}} &= X^TW_{ij}X
\end{align*}
Thus the Hessian of the Log-Likelihood is the block matrix H with
blocks $H_{ij} = W_{ij}$, where this is the Hessian of the flattened vector $\theta$
made by combining all $\theta^{(1)}, \ldots, \theta^{(K - 1)}$. Note that the Gradient
of this flattened vector is the vector created by stacking the columns of the Jacobian.

\section*{Optimization}
The objective function will be maximized via the Newton-Raphson Method. Reusing work from the previous write up,
the update rule is given by:
\begin{align*}
\theta^{(new)} &= \theta^{(old)} + H^{-1}(\nabla_\theta(\mathcal{L})(\theta^{(old)}))
\end{align*}
Where H represents the full hessian matrix and $\nabla_\theta(\mathcal{L})$ is the
flattened Jacobian where each column of the Jacobian is stacked (So that the gradients
of each class parameter are kept together). If we let $g = \nabla_\theta(\mathcal{L})(\theta^{(old)})$
We can simplify the computations in each step of the method by solving for $\Delta \theta = \theta^{(new)} - \theta^{(old)}$.
Where $H(\Delta \theta) = g$. $H$ is a semi-positive definite matrix. If we apply some regularization we can ensure
that $H$ is a positive definite matrix, and so can use the conjugate gradient method to
solve for $\Delta \theta$. This method leverages the property that given $u, v \in \mathbb{R}^{m}$ ($m := d\times(C-1)$),
$u^THv$ defines an inner product $\langle u, v \rangle_H := u^THv$, a fact that is easy to check. By 
Gram-Schmidt we can always find a basis of orthogonal vectors $P_1, \ldots, P_m$ for the inner product space
given by $\mathbb{R}^m$ with respect to this inner product. Thus, we can express $\Delta \theta$ as a linear
combination of these vectors with coefficients $\alpha_1, \ldots \alpha_m$. Thus:
\begin{align*}
H(\Delta \theta) &= \displaystyle H(\sum_{i=1}^{m}\alpha_ip_i) \\
g &= \sum_{i=1}^{m}\alpha_iHp_i
\end{align*}
Now left multiplying by taking the dot product with $p_k$ we have:
\begin{align*}
    p_k^Tg &= \displaystyle p_k^T\sum_{i=1}^{m}\alpha_iHp_i \\
    &= \sum_{i=1}^{m}\alpha_ip_k^THp_i \\
    &= \sum_{i=1}^{m}\alpha_i\langle p_k, p_i \rangle_H
\end{align*}
By the orthonormality of the basis we have:
\begin{align*}
    p_k^Tg &= \alpha_k\langle p_k, p_k \rangle \\
    \alpha_k &= \frac{p_k^Tg}{\langle p_k, p_k \rangle_H}
\end{align*}
This gives us an algorithm for computing the coefficient of a basis vector.
We can combine this with the Gram-Schmidt algorithm for computing this orthogonal
basis and recover the coefficients at each step.


This can save us a lot of time as H is only used to compute the inner product,
but since H is a block matrix and v is a block vector we can use this to compute $Hv$
without ever constructing $H$, saving a lot of computing time. To do this consider
the block vector $Hv$ where $H$ and $v$ are viewed as a block matrix and a block vector
respectively. You can think of $v$ as a magtrix with each column $i$ being an arbitrary 
parameter vector for class $i$. By the block matrix-vector multiplication rule we have:
\begin{align*}
    (Hv)_i &= \displaystyle \sum_{j = 1}^{C-1} H_{ij}v_j \\
    &= \sum_{j = 1}^{C-1} X^TW_{ij}Xv_j \\
    &= X^T\sum_{j = 1}^{C-1} W_{ij}Xv_j
\end{align*}
Fix $P$ to be the matrix with its r-th row containing the probability vector
of the r-th training sample. Now consider first $Xv_j$:
\begin{align*}
    Xv_j &= \begin{bmatrix}
        X_1^Tv_j \\ \vdots \\ X_n^Tv_j
    \end{bmatrix}
\end{align*}
Since $W_{ij}$ is a diagonal matrix, multiplying $Xv_j$ by it corresponds to
elementwise scaling by the diagonal elements of $W_{ij}$.
\begin{align*}
    Xv_j &= \begin{bmatrix}
        X_1^Tv_j \\ \vdots \\ X_n^Tv_j
    \end{bmatrix}
\end{align*}


\end{document}